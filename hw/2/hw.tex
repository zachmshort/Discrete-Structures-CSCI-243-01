\documentclass[11pt]{article}

\usepackage{exscale}
\usepackage{graphicx}
\usepackage{amsmath}
\usepackage{latexsym}
\usepackage{times,mathptm}
\usepackage{epsfig}
\usepackage{amsmath}
\usepackage{enumitem}
\usepackage{mathpazo}

\textwidth 7.5truein          
\textheight 9.5truein
\oddsidemargin 0.0in
\topmargin -.9in
\thispagestyle{empty} 

\parindent 0pt          
\parskip 0pt

\begin{document}

{\bf Name: Zachary Short}
\hspace{10.75cm}
{\bf Section: 1}

\begin{large}
\centerline {\bf Computer Science 243}
\centerline {\bf Spring 2025}
\centerline {\bf Homework 2}
\centerline{\bf Dr. Davis}
\vskip .7cm

\bf Due: beginning of class, Monday, 2/10/25 \hfill Points: 50
\end{large}
\vskip 0.5cm
Answer the following questions and show your work.  Your final submission must be completely your own work.
\begin{enumerate}
\item {[9 points each]}  Prove the following using logical equivalences (Chapter 1-I, Slides 40-42, or Tables 6 – 8 on pp. 27-28).  
Label each equivalence used. 

\begin{enumerate}[label=\alph*., itemsep=0mm] 
\item $\neg ((p \rightarrow \neg q) \equiv p \vee \neg q$ 
\item  $\neg ((p \rightarrow \neg q)\equiv \neg ((p \rightarrow \neg q) \wedge q)$ 
\item \hspace{2cm} $\equiv \neg ((\neg p \vee \neg q) \wedge q)$ Implication Rule
\item \hspace{2cm} $\equiv (\neg(\neg p \vee \neg q) \vee \neg q)$  De Morgan's
\item \hspace{2cm} $\equiv ((p \wedge q) \vee \neg q)$ De Morgan's
\item \hspace{2cm} $\equiv (p \vee \neg q)  \wedge (\neg q \vee q)$ Distributive
\item \hspace{2cm} $\equiv (p \vee \neg q)  \wedge T$ Negation
\item \hspace{2cm} $\equiv p \vee \neg q$ Identity

\item $(p \rightarrow (q \rightarrow p)) \wedge ((p \wedge \neg q) \rightarrow p) \equiv$ 
{\bf T}
\item $(p \rightarrow (q \rightarrow p)) \wedge ((p \wedge \neg q) \rightarrow p) \equiv (p \rightarrow (p \rightarrow (q \rightarrow p)) \wedge ((p \wedge \neg q) \rightarrow p)  $ 
\item \hspace{5.55cm}$ \equiv (p \rightarrow (p \rightarrow (\neg q \vee  p)) \wedge ((p \wedge \neg q) \rightarrow p)  $ Implication
\item \hspace{5.55cm}$ \equiv (p \rightarrow (\neg p \vee (\neg q \vee p)) \wedge ((p \wedge \neg q) \rightarrow p)  $ Implication
\item \hspace{5.55cm}$ \equiv (p \rightarrow (\neg q \vee (\neg p \vee p)) \wedge ((p \wedge \neg q) \rightarrow p)  $ Associative
\item \hspace{5.55cm}$ \equiv (p \rightarrow (\neg q \vee T) \wedge ((p \wedge \neg q) \rightarrow p)  $ Negation
\item \hspace{5.55cm}$ \equiv p \rightarrow T \wedge ((p \wedge \neg q) \rightarrow p)  $ Domination
\item \hspace{5.55cm}$ \equiv T \wedge ((p \wedge \neg q) \rightarrow p)  $ Tautology
\item \hspace{5.55cm}$ \equiv T \wedge (\neg(p \wedge \neg q) \vee p)  $ Implication
\item \hspace{5.55cm}$ \equiv T \wedge ((\neg p \vee q) \vee p)  $ De Morgan's
\item \hspace{5.55cm}$ \equiv T \wedge ((\neg p \vee p) \vee q)  $ Associative
\item \hspace{5.55cm}$ \equiv T \wedge (T \vee q)  $ Negation
\item \hspace{5.55cm}$ \equiv T \wedge T  $ Domination
\item \hspace{5.55cm}$ \equiv T  $ 
\end{enumerate}

\vskip 0.5cm
\item {[3 points each]}  Consider the following, where the domain for $x$ and $y$ is $\{-1, 1\}$:\\
\\ \hspace*{2cm} $p(x,y): x = |y|$
\\ \hspace*{2cm} $q(x): x < 0$ \\
\\ Based on the above assignments, state the truth value for the each of following, along with your reasoning 
(i.e., list the values of $x$ and/or $y$ that make the statement true or false):
\begin{enumerate}[label=\alph*.]
\itemsep-0mm
\item $\forall x \exists y$ $p(x, y)$ Not true, if x = -1 and y = 1, $-1 \neq 1$
\item $\exists y \forall x$ $p(y, x)$ True, if y is 1, then y would be equal to the entire range of x, 1 = 1 and 1 = $\mid -1 \mid$.
\item $\forall x \forall y$ $(p(x, y) \rightarrow q(x))$ Not true, $p(1,1) \rightarrow q(1) $ fails.
\item $\exists x \forall y$ $(q(x) \rightarrow p(x, y))$ True, $q(-1) \rightarrow p(-1,1) $ an$q(-1) \rightarrow p(-1,-1) $ are true, covering the full range of y.
\end{enumerate}

\vskip 0.5cm
\item {[3 points each]}  Given a domain of the integers for $x$ and the negative integers for $y$, 
state the truth value of each of the following statements, along with your reasoning.

\begin{enumerate}[label=\alph*.]
\itemsep-0mm
\item $\forall x \exists y$ $(x^2 = y)$ False, any number squared is positive. 
\item $\forall x \exists y$ $(x = y^2)$ False, if you square root both sides the expression becomes $\sqrt{x} = y$, if x is a negative, this fails because an imaginary number cannot equal a negative integer.
\item $\exists x \forall y$ $(y^x = 1)$ True, if x = 0, any number to the power of 0 equals 1.
\item $\forall x$ $(x \neq 0 \rightarrow$ $\exists y$ $(xy > 1))$ False, if x is positive, there's no y you could multiply it by to make it positive because all y values are negative. 
\end{enumerate}

\vskip 0.5cm

% \newpage

\item {[2 points each]}  Given a domain of all people and the following predicates:
\\ \hspace*{3cm} {$S(x)$}: $x$ is a student.
\\ \hspace*{3cm} {$W(x)$}: $x$ lives in Williamsburg.
\\ \hspace*{3cm} {$E(x,y)$}: $x$ emails $y$.

translate the following English expressions into first-order logic statements:

\begin{enumerate}[label=\alph*.]
\itemsep-0mm
\item Some people living in Williamsburg are students. 
$\exists x (S(x) \wedge W(x)$
\item 
All students who live in Williamsburg email at least one other student.\subsection{$\forall x  ((S(x) \wedge W(x))  \rightarrow \exists y(S(y)\wedge E(x,y))$} 
\item Nobody emails every student. $\neg \exists x \forall y (S(y) \rightarrow E(x,y))$
\item No students who live in Williamsburg email themselves. $\neg \exists x ((S(x) \wedge W(x)) \rightarrow E(x,x))$
\end{enumerate}
\end{enumerate}
\end{document}